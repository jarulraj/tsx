\section{Optimistic Concurrency Control in Haswell} \label{sec:tm}

A simple mechanism that can be used to provide atomicity and isolation is
coarse-grained locks. This allows at least a limited amount of safe concurrency.
However, this is a pessimistic concurrency control protocol that targets
high-contention workloads. Read accesses also need to incur the locking overhead
in this scheme.

For low contention workloads (e.g, read-dominant workloads), an optimistic
concurrency control scheme generally provides better performance, as the
validation is postponed to the end of the transaction. The concurrency control
protocol is essentially not exposed to the user, which allows non-conflicting
accesses to avoid the overhead of locking.  Optimistic concurrency control
mechanisms still rely on locking while speculatively executed transactions are
being committed, but they move this functionality to the language runtime layer
in the case of software transactional memory schemes or to the hardware layer in
the case of HTM schemes. The programmer is presented only the higher-level
abstraction of transactions.

HTM simplifies concurrent programming by providing support for atomic execution
of a set of load and store instructions by using transactional caches. Intel's
new Haswell processors provide an HTM implementation that allows non-conflicting
transactions to elide locks entirely, serializing only conflicting transactions.

The Haswell processors implement HTM in the form of instruction set extensions
known as Transaction Synchronization Extensions (TSX). TSX supports two
interfaces \citep{tsx-intro}: \\

\begin{itemize} 
\item \textbf{Hardware Lock Elision (HLE):} \\ This is a backwards-compatible
  extension which allows specifying transaction regions via \textrm{XACQUIRE}
  and \textrm{XRELEASE} instruction prefixes. HLE is compatible with the
  lock-based programming model. In HLE, when the transaction execution fails,
  the TM implementation simply reverts to normal lock semantics during
  re-execution. \\
\item \textbf{Restricted Transactional Memory (RTM):} \\ This is a more flexible
  extension, albeit not a backward compatible one, which allows specification of
  transactions using \textrm{XBEGIN, XEND and XABORT} instructions. It allows
  the programmer to specify the fallback path for transaction failures. \\
\end{itemize}

In the RTM interface, the conflict resolution mechanism is flexible -- the
programmer can define the fallback mechanism to resolve conflicts. For instance,
a group sum operation using RTM interface in Intel Haswell is shown in
\Cref{fig:rtm}.  In contrast, the HLE interface relies on a more conservative
policy in the case of conflicts: the processor simply grabs the lock that had
been elided. We evaluated both interfaces in this project.

\lstset{basicstyle=\ttfamily\fontsize{9}{10}\selectfont,
morekeywords={if,else,end}, numbers=left} \begin{figure}
    \parbox[t]{0.45\textwidth}{\lstinputlisting{figure/rtm.c}} \caption{Group
        sum operation using RTM interface. A mutex is obtained in fallback path
to avoid data race with the normal transaction execution path.} \label{fig:rtm}
\end{figure}
