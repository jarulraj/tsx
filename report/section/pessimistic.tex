\section{Pessimistic Concurrency Control} \label{sec:pessimistic}

In order to have a baseline against which to compare the performance of HTM, we will be implementing two different forms of traditional, software based pessimistic concurrency control using locks - a lock manager and spin locks.

\subsection{Lock Manager}

Traditional locking schemes make use of a lock manager, which arbitrates access to record in the key-value store by granting locks to requesting transactions on a per key-value pair basis. A lock manager consists of a lock table hashed on keys. Each lock record contains a mode, either read, write, or free, and a list of transactions waiting to acquire the lock. A transaction may only access a given key-value pair once it has requested and been granted the appropriate lock, preventing conflicts.

When a process requests a lock, the lock manager looks up the lock's mode, and if it is free, the lock is  granted. It is also granted if the lock's mode is read and the request is for read access. If either the mode is write or the mode is not free and the request is for write access, then the requesting process is put on the end of the lock's waiting queue. This is because only writes can cause conflicts - any number of transactions can execute concurrent reads without affecting each other. When the transaction currently holding the lock releases it, the lock manager grants the lock to the transaction at the head of the waiting queue.

Lock managers have several options for dealing with deadlocks, either by preventing them or by dynamically detecting and breaking them. A simple approach, which we will implement, is to enforce an ordering in which a transaction may request locks, ensuring that no circular dependencies between transactions waiting on locks may occur. This works well under our assumption that the read and write sets of transactions are static, but can be difficult if the read and write sets cannot always be known at the start of the transactions.

\subsection{Spin Locks}

In a spin lock scheme, each key-value pair has an associated memory word representing its lock. Locks are acquired using an atomic test-and-set primitive. If a transaction attempts a test-and-set but it fails because the lock is already held, the transaction "spins", or repeatedly attempts to grab the lock until it eventually succeeds.

Because there is no global list of waiting transactions, there are fewer options for deadlock detection. As with the lock manager, we will implement deadlock prevention by enforcing an order in which locks must be obtained. In a system where you cannot assume static read and write sets, deadlock with spin locks can be avoided by aborting waiting transactions after a timeout, on the assumption that waiting for a lock for an extended amount of time is an indication that deadlock has occurred.

Spin locks are much simpler to implement than the more complicated lock manager, and by avoiding the extra work of going through the manager they can potentially be faster. However, the obvious disadvantage of spin locks is that they require busy waiting - when a transaction spins while waiting for a lock, it is using CPU time without really doing any useful work. This is primarily a problem in systems with high contention, where it is likely that multiple transactions will want to access the same key-value pair at the same time. As a result, we expect spin locks, like optimistic concurrency control, to perform better in systems with low contention.