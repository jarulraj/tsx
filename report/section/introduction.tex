\section{Introduction} \label{sec:intro}

Transactional memory (TM) is a mechanism to provide
fine-grained read/write access to multiple memory words in a lock-free manner
\citep{Herlihy93}. Several implementations of TM have been realized in hardware
and software. For instance, the load linked/store conditional primitive available
in modern processors is a special form of TM, where access is regulated to only
one memory word. Hardware implementations generally rely on extensions to the
cache-coherence protocol. Software implementations on the other hand provide
transactional interfaces by using basic hardware primitives like atomic compare and
swap (CAS).

In recent Intel Haswell processors, transactional memory support has been added
in the form of instruction set extensions known as Transaction Synchronization
Extensions (TSX). TSX supports two interfaces \citep{tsx-intro} : \\


\begin{itemize} 
\item \textbf{Hardware Lock Elision (HLE)} \\ This is a backward
compatible extension which allows specification of transaction regions using
\textrm{XACQUIRE} and \textrm{XRELEASE} prefixes. It is compatible with the lock-based programming
model. In HLE, when the transaction execution fails, the TM implementation
simply skips the transactional semantics during re-execution. \\ 

\item \textbf{Restricted Transactional Memory (RTM)} \\ This is a more
flexible albeit not backward compatibile extension which allows specification of
transactions using \textrm{XBEGIN, XEND and XABORT} instructions. It allows the
programmer to specify the fallback path when the transaction fails.  \\ 
\end{itemize}

These hardware primitives can be used to implement concurrency control 
required for synchronizing accesses to data structures. In this project,
we plan to focus specifically on \textit{multi-key transactions} on a
\textit{key-value store}. In \Cref{sec:tm},
we describe the problem of synchronizing multi-key transactions on a 
key-value store using hardware TM support.
We then present the traditional pessimistic concurrency control protocols 
we plan to use for comparison in \Cref{sec:pessimistic}. Finally, we
define the goals of this project and our plan to split the workload
in \Cref{sec:plan}.


