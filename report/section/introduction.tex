\section{Introduction} \label{sec:intro}

Transactional memory (TM) has been a relatively esoteric approach to provide
fine-grained read/write access to multiple memory words in a lock-free manner
\citep{Herlihy93}. Sevaral implementations of TM have been realized in hardware
or software. For instance, the load linked/store conditional primitive available
in modern processors is a special form of TM, where access is regulated to only
one memory word. Hardware implementations generally rely on extensions to the
cache-coherence protocol. Software implementations on the other hand provide
interfaces that build upon the basic hardware primitives like atomic compare and
swap (CAS).

In recent Intel Haswell processors, transactional memory support has been added
in the form of instruction set extensions known as Transaction Synchronization
Extensions (TSX).  TSX supports two interfaces \citep{tsx-intro} : \\


\begin{itemize} 
\item \textbf{Hardware Lock Elision (HLE)} \\ This is a backward
compatible extension which allows specification of transaction regions using
XACQUIRE and XRELEASE prefixes. It is compatible with the lock-based programming
model. \\ 

\item \textbf{Restricted Transactional Memory (RTM)} \\ This is a more
flexible albeit not backward compatibile extension which allows specification of
transactions using XBEGIN, XEND and XABORT instructions. It allows the
programmer to define the fallback path when the transaction fails unlike HLE
wherein the transactional semantics are just skipped after the first transaction
abort. \\ 
\end{itemize}

These hardware primitives can be used to implement the concurrency control 
required for synchronizing accesses to data structures. In this project,
we plan to focus on multi-key transactions on a key-value store. In \Cref{sec:tm},
we formulate the problem of synchronizing accesses to a key-value store, especially
in the context of hardware transactional memory support.